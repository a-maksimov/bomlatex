%%% eskdx
\renewcommand{\ESKDfontShape}{\upshape}

%%% шрифты листа регистрации изменений
\renewcommand{\ESKDfontTabHead}{\scriptsize}
\renewcommand{\ESKDfontTabBody}{\scriptsize}

%%% стиль заголовков
\ESKDsectStyle{section}{\normalfont}
\ESKDsectStyle{subsection}{\normalfont}
\ESKDsectStyle{subsubsection}{\normalfont}
\ESKDsectSkip{section}{8mm}{8mm}
\ESKDsectSkip{subsection}{8mm}{8mm}
\ESKDsectSkip{subsubsection}{15mm}{15mm}

%%% содержание
% содержание без выделений и отступов
\makeatletter
	\renewcommand{\l@section}{\@dottedtocline{1}{0em}{1.25em}} % секции
	\renewcommand{\l@subsection}{\@dottedtocline{2}{0em}{2em}} % подсекции
\makeatother

%%% таблицы
% команды для интервалов до линий в tabluar и др.
\newcommand\T{\rule{0pt}{2.6ex}} % сверху
\newcommand\B{\rule[-1.2ex]{0pt}{0pt}} % снизу
% интервал до линий в tabu не менее:
\tabulinesep=1.5mm 
% счетчик строк
\newcounter{rowcount}
\newcommand\rownumber{\stepcounter{rowcount}\arabic{rowcount}}
% счетчик столбцов
\newcounter{colcount}
\newcommand\colnumber{\stepcounter{colcount}\arabic{colcount}}
% формат номера для второй шапки longtabu или longtable
\DeclareCaptionLabelFormat{continued}{Продолжение таблицы~#2} 

%%% рисунки
% вертикальная отбивка между подписью и содержимым рисунка
\captionsetup[figure]{skip=10pt}
% подпись рисунка по ГОСТ 2.105, 4.3.1
\DeclareCaptionLabelSeparator*{emdash}{~--- }
\captionsetup[figure]{labelsep=emdash}

%%% списки
% стиль списков с абзацным отступом
\setlist{nolistsep, leftmargin=0cm, itemindent=2.2cm} % стиль списков 1 уровня
\setlist[2]{leftmargin=17mm} % стиль списков 2 уровня
% новый стиль списка для пояснений элементов схем
\newlist{picdescription}{enumerate}{1}
\setlist[picdescription]{%
	before={\vspace{10pt}\small},%
	font=\small,%
	label={\arabic* ---},%
	ref={\arabic*},%
	leftmargin=2cm,%
	itemindent=0cm,%
	rightmargin=1cm,%
	labelsep=\widthof{\ },%
	nolistsep}
% нумерация списков:
%% кириллические буквы для 1 уровня
\makeatletter 
	\AddEnumerateCounter{\asbuk}{\@asbuk}{м)}
\makeatother
\renewcommand{\labelenumi}{\asbuk{enumi})}
%% арабские цифры для 2 уровня
\renewcommand{\labelitemii}{\arabic{enumii})}
% используем короткое тире (endash) для ненумерованных списков (ГОСТ 2.105-95, пункт 4.1.7, требует дефиса, но так лучше смотрится)
\renewcommand{\labelitemi}{\normalfont\bfseries{--}}

% размещение float объекта сверху float-only страницы, а не по середине
\makeatletter
	\setlength{\@fptop}{0pt}
	\setlength{\@fpbot}{0pt plus 1fil}
\makeatother

% исправление ситуации с обновленным babel-russian, в котором больше не поддерживается \No (в новых документах следует использовать \textnumero)
\newcommand{\No}{\textnumero}